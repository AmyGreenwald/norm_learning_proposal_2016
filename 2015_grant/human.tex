
\subsection{}

Human interactions are governed by social norms; for human--machine
interactions and collaborations to be successful, machines must abide
by these norms as well. But what does it mean to abide by norms?
Simple rule-following machines will not instill trust in human
collaborators, because they would be insensitive to variations in
context.\commenta{robustness}  Rather, machines must exhibit a norm capacity that
approximates the human norm capacity, which seems to have such unique
properties as context sensitivity, hierarchical relations, continuous
updating, conflict resolution, and so on.  Surprisingly, however,
research into the cognitive properties of human norm capacity has been
rather limited. Similarly, but less surprisingly, efforts to design
and implement even modest norm capacities in computational systems
have been sparse.

We therefore take a two-pronged approach: We must conduct empirical
research on how humans learn, represent, activate, and implement
social norms; and we must design formalisms and algorithms that allow
machines (especially robots) to learn, represent, activate, and
implement social norms.  The empirical research on human norm capacity
plays a dual role in this project: it provides the necessary knowledge
to build computational {\em models} of human norm-guided behavior (so
robots can better understand and predict human behavior); and it
informs and sets the standards for building computational {\em
  mechanisms} of machine norm capacities (so robots can take actions
that comply with the relevant norms).

%%% CUT FROM ELSEWHERE !!!

Thus, building effective robotic teammates, and machines as partners
more generally, critically requires us to develop computational
mechanisms for social norm processing.  But we can do so only if we
have substantial knowledge of how human norm systems work, for robot
norms and their cognitive properties will have to match those of
humans for the robot to be socially acceptable.  Surprisingly little
research is available on the cognitive and computational properties of
human norm systems, so we must make substantial advances in the
cognitive science of human norms and, in parallel, we must make
equally substantial advances in designing algorithms that implement
such norms.

