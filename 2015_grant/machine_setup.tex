
%\vspace{\up}
\paragraph{Simulation Experiments}
\label{sec:qlearning}

We carried out a set of simulation experiments with \Q-learning in
the grid games presented.
%
For each grid game, we conducted 50 independent runs in which two
\Q-learners faced off.  
%Runs lasted for 30,000 rounds each.  
The agents' value functions were optimistically initialized with a
value of 40 for all states and they used Boltzmann exploration with a
temperature of $0.5$.  The discount factor was set to $0.9$, and the
learning rate to $0.01$.  To ensure that the state space was
adequately explored, any state that is reachable from the initial
state had some probability of being selected as the starting position
for a round.  Once either agent reached its goal (or 100 moves were
taken), the round was terminated.  There were no step costs beyond discounting, and
rewards were 50 for reaching a goal.

We denote the outcome of a round using a pair of letters, where {\bf G} 
means the agent reached the goal and {\bf N} means the agent did not
reach the goal. The first letter in the pair represents the agent's
own outcome and the second represents its opponent's outcome. For
example, {\bf GN} is used to denote that the agent reaches its goal
while its opponent does not.

As two \Q-learning algorithms are not guaranteed to converge in
self-play, we arbitrarily stopped the learning after 30,000 rounds,
and checked the strategies learned.  In spite of \Q-learning not
explicitly seeking outcomes with high social welfare, it very reliably
identified cooperative strategies leading to {\bf GG} outcomes.
% zzz do we provide results to reference?

Only the No Compromise game posed a challenge to the \Q-learners.
There, they tend to thrash about, finding a pair of strategies that
work well together only to eventually discover that one of the agents
has an opportunity to defect.  The defection is destabilizing, so a
new search begins for strategies that work well together.  This result
is not altogether unsurprising, because No Compromise is the only game
in our testbed that does not possess a pair of CD strategies that
constitute a Nash equilibrium.

%Sometimes, \Q-learning finds strategies that are cooperative and stable,
%but these strategies are brittle and would not be successful against
%agents with other strategies.

