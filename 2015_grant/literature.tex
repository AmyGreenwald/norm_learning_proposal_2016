
\section{Related Work}
\label{sec:related}

We are proposing to formalize social norms in artificial agents.  Our
plan, ultimately, is to test our representation and learning
algorithms by embodying them in structured virtual environments where
they interact with humans.  To begin, we provide a short review of the
(vast) body of literature on understanding of human behavior in
similar experimental environments.  Specifically, we discuss related
work in behavioral economics and cognitive psychology on how people
make decisions when their rewards depend not only on the state of the
environment, but also on the decisions of other agents in that
environment.

Behavioral economists have studied how different structural and
contextual factors, such as reward structure, communication, and group
membership~\cite{Camerer:2003}, affect the emergence of coordination
on different equilibria in the actions chosen by each agent after
repeated play. For example, in the Ultimatum Game~\cite{guth82}, there
are two participants, a ``proposer'' and a ``responder''. The proposer
makes an offer to the responder as to how to split some pile of
money. If the responder accepts the offer, the participants are
rewarded based on this split. If the responder rejects the offer,
neither receives a reward. According to the normative prediction of
expected utility theory (EUT)~\cite{vonneumann44}, the proposer should
offer the smallest increment possible to the responder, and the
responder should accept any positive offer. However, many studies have
found that both of these normative predictions are violated: Proposers
offer more money to responders than predicted (more even splits of the
money) and responders reject non-zero offers that they perceive to be
unfair.

To examine why the proposers offer splits closer to even---perhaps
because of altruism, or fear of rejection---Forsythe et
al.~\cite{forsythe94} tested participants in the Dictator
Game~\cite{kahneman86}, where the proposer decides on how to split
some amount of money and the responder is forced to accept
it. Although splits are significantly more in favor of the proposer in
this game, the responder is still given more money than the normative
theory would predict (i.e., the responder is given more than
nothing). They found that the dictator gives, on average, 25\% of the
total sum to the responder, a finding that has been replicated
multiple times. Although scientists continue to disagree about why
people still share their money in this situation,
%and the amount shared varies depending on culture and other variables, 
one viable explanation is that fairness is a nearly universal norm,
especially in industrialized cultures~\cite{henrich05}.

In addition to exploring the nature of established norms, such as
fairness, behavioral economists have also investigated if, when, and
how norms emerge over repeated interactions between two agents whose
joint best-interest requires trust. Specifically, the Prisoners'
Dilemma~\cite{luce57} is a symmetric game where each agent chooses to
either ``cooperate'' or ``defect''. If both agents cooperate, they
both receive higher payoffs than if both agents defect. However, if
one agent cooperates and the other agent defects, the defecting agent
receives more payoff than when both agents cooperate (and the
cooperating agent receives even less payoff than when both agents
defect). Thus, each agent has an incentive to defect, even when they
believe the other agent will cooperate. Indeed, both agents defecting
is the unique Nash equilibrium. Nonetheless, people cooperate more
than predicted by the Nash equilibrium across a wide range of
conditions, including but not limited to the ``one-shot''
case~\cite{rapoport88}, where each pair only plays PD only once, real
contestants on game shows~\cite{list06}, and especially when the same
pair of subjects plays the game repeatedly~\cite{rand13,sally95}.

Another game used to explore the nature of coordination in the face of
conflicting goals is Battle of the Sexes (BOS)~\cite{luce57}. In
traditional versions of BOS, two agents choose one of two options
simultaneously. Although each agent prefers a different option, the
two agents prefer to choose the same option rather than different options.
%Perceived timing (simultaneous vs.\ sequential moves) has been shown
%to have strong effects on the decisions people make in this game. 
%For example,
However, Cooper et al.\cite{cooper94} found that letting
one person choose first (with both participants knowing who
chose first, but not what they chose) increased the odds
that players would match. In particular, when participants
chose simultaneously, they {\em mismatched} 60\% of the time,
but when one chose before the other, they settled on the 
preferred equilibrium for the first acting agent 62\% of the time.
Similar results of decision sequentiality affecting decisions
was also found in other work~\cite{rapoport97,ho96}.


%In one version of BOS~\cite{cooper94}, one participant told their
%decision to the experimenter (but not to the other participant) before
%the other participant, and both participants were aware of this
%fact. From a normative (EUT) perspective, this should not affect which
%equilibrium the participants agree on, but it does in practice. The
%later-moving player seems to take into account the fact that the
%earlier-moving player moves first. Normally, in BOS, when participants
%move simultaneously and both know they are moving simultaneously,
%participants mismatch 60\% of the time. But if player $A$ knows that
%player $B$ is going to move before them, the dyad mismatches only 34\%
%of the time and settles on the preferred equilibrium for $B$ 62\% of
%the time. Once again, this is without $A$ having any knowledge of
%$B$'s move~\cite{cooper94}.
%%
%Further, in a $k$-person simultaneous analog,
%Rapoport~\cite{rapoport97} found that participants settle on the first
%person's preferred equilibrium much more often than on any others. In
%both of these experiments, the environment being perceived as
%sequential results in different behavior from when the environment is
%perceived to be simultaneous. Divergent behavior in
%formally-equivalent games is not isolated to timing effects in
%BOS. Another case of such differences is that participants act differently
%in games depending in on whether they are presented in extensive- or
%normal-form (e.g., Ho \& Weigelt~\cite{ho96}).

Although exploring how people can settle into different equilibria in
formally-equivalent games is a relatively underdeveloped topic in
behavioral economics (some exceptions
include~\cite{cooper03,devetag08,diguida13}), there is a long history
of examining how people react differently to the same stimulus
depending on its framing and other contextual and ecological factors
in cognitive psychology. To explain how people arrive at different
decisions given formally-equivalent input, psychologists posit and
investigate how framing results in participants encoding the current
situation using different representations, which then result in
different decisions~\cite{austerweil13,chomsky59}. In fact, some
psychologists believe that it is impossible to study how people act
unless they are in realistic environments, because cognition is
fundamentally tied to its embodiment in an environment and the
interactive nature of real-life
activities~\cite{hutchins10,wilson02}. Thus, to try to unravel how and
why norms emerge in various decision-making situations, it is
necessary to embody agents in as realistic an environment as possible,
while maintaining the capability to control experimental factors.

As a step towards studying how structural and contextual factors can
affect the emergence of norms in games, we propose to conduct our
study in a sequential \mydef{grid game} framework, where each agent
takes actions to move itself through a grid to get to its goal.  The
strategy (also called policy) space in these games is much richer than
that of the aforementioned games (BOS, PD, and the Ultimatum and
Dictator games), while at the same time mimicking some of their
features: e.g., a decision to cooperate or defect; a notion of
fairness, etc..  Specifically, in all the games we describe in this
proposal, the environment is constructed such that the agents cannot
reach their goals without coordinating their behavior.  In this way,
our games are designed to capture the rich back and forth negotiations
that must take place when agents successfully coordinate in the real
world, while at the same maintaining computational tractability
(hence, the restriction to grid games).

