
\section{Results from Prior NSF Support}

\paragraph{Intellectual Merit}

Greenwald is currently funded by NSF (RI: Small--1217761, 2012--2016,
\$450k) to build artificial agents that assist humans with decision
making in information-rich and time-critical environments like online
markets.  This work is ongoing, but some preliminary publications
include~\cite{tada:jack,seqauc:nips,efgs:rldm,tada:quibids}.
%
Previously, she was funded to develop ``Methods of Empirical Mechanism
Design (EMD)'' (CCF: Medium--0905234, 2009--2012, \$850k, with Mike
Wellman).  This project expanded the scope of mechanism design beyond
the small-scale, stylized, or idealized domains most previous work was
limited to.  Publications include~\cite{poi:aamas,poi:ec,simspsb:uai}.
%
Before that, she received two prior NSF awards, a PECASE CAREER grant,
``Computational Social Choice Theory'' (IIS: Career--0133689,
2002--2007, \$375k), and
%a second award, 
``Efficient Link Analysis'' (IIS: Small--0534586, 2005--2008,
\$363k), which focused on ranking web pages and other social networks
with hierarchical structure.

\comment{
(iii)	Results from Prior NSF Support

If any PI or co-PI identified on the project has received NSF funding in the past five years, information on the award(s) is required. Each PI and co-PI who has received more than one award(excluding amendments) must report on the award most closely related to the proposal. The following information must be provided:

(a)	the NSF award number, amount and period of support;

(b)	the title of the project;

(c)	a summary of the results of the completed work, including, for a research project, any contribution to the development of human resources in science and engineering;

(d)	publications resulting from the NSF award;

(e)	a brief description of available data, samples, physical collections and other related research products not described elsewhere; and

(f)	if the proposal is for renewed support, a description of the relation of the completed work to the proposed work.

Reviewers will be asked to comment on the quality of the prior work described in this section of the proposal. Please note that the proposal may contain up to five pages to describe the results. Results may be summarized in fewer than five pages, which would give the balance of the 15 pages for the Project Description.
}

Littman is a co-PI on ``RI: Medium: Collaborative Research: Teaching
Computers to Follow Verbal Instructions'' (No.\ 1065195, 9/11--8/14,
\$704K) and ``RI: Small: Collaborative Research: Speeding Up Learning
through Modeling the Pragmatics of Training'' (No.\ 1319305,
10/13--9/15, \$440k). These proposals developed autonomous agents that
extract preferences from people via verbal interaction and reward
feedback~\cite{loftin14b,macglashan15,macglashan15b}.

\paragraph{Broader Impact}

Broader impacts of our past work include undergraduate and graduate
student training, design and evaluation of learning modules for
outreach programs Learning Exchange and Artemis, and organizing two
major computer science conferences (Littman)~\cite{desjardins13}.
We also published novel benchmarks for evaluating grounded language
learning programs (Littman), and developed a simulation platform for
empirical game-theoretic analysis (Greenwald).

Greenwald also had an additional NSF award (EAGER: 1059570,
2010--2012, \$90k) which supported the expansion and evaluation of the
Artemis Project, a free, summer program in which Brown computer
science women teach computer science to rising 9th grade girls.  This
program has the dual effect of empowering Brown women, while at the
same time, exposing girls to computer science.  With this money, she
successfully expanded Artemis to BU, and ran pilot programs at
Columbia and UMBC.

Greenwald also has \$10k in funding from the Tides Foundation, through
the IgniteCS progam, for a project entitled Bringing Computer Science
Education to Providence Public Schools.

