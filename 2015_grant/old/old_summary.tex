
\centerline{\Large \bf Representing and Learning Norms:}
\centerline{\large \bf A Key to Human-Machine Collaboration}


\paragraph{Intellectual Merit}

Human-machine collaboration will never be achieved at scale unless our
machines can be counted on to regularly engage in productive social
interactions with us.  Working towards the ultimate goal of machines
as our partners, we propose to focus on one fundamental driver of
social interactions: \emph{social norms}.  We claim that we cannot
we make progress toward designing machines that are genuine
partners---machines that learn from and teach humans, assist humans,
collaborate with humans, etc.---unless we make progress toward
building machines that can represent and learn norms.

Norms guide and constrain every cultural behavior, from eating and
dressing to speaking and working.  Indeed, humans expect one another
to grasp and abide by a dizzying number of norms, so much so that
failure to follow norms can result in disastrous consequences (e.g., a
British tourist driving on the left side of the road in Australia).
If humans are to interact with machines in genuinely collaborative
situations, they will expect those machines to similarly grasp and
abide by the relevant social norms.

The present project takes a first step toward building the social
foundations of human-machine collaboration by focusing on the
normative context in which such collaboration is inevitably embedded.
In particular, we will pursue two aims:
%
(i)~to develop computational formalisms required for representing
social norms in AI agents; and (ii)~to provide a proof of concept that
norms are computationally learnable by AI agents by developing and
evaluating prototype algorithms for norm learning in multi-agent
settings that involve both humans and machines.


\paragraph{Broader Impact}


%\paragraph{Collaboration Plan}
%\paragraph{Students' Experiences}
%\paragraph{Unique Opportunity}


Figure~\ref{fig:visionobjective} indicates how the broad vision of
machines as partners can be substantially advanced by pursuing the
long-term objective of implementing social-cognitive capacities in
machines.  Importantly, it also indicates how the proposed short-term
project on norm capacity initiates a powerful and necessary first step
in this direction, promising novel scientific insight into norm
representation, and beginning to develop algorithms that implement
such capacity, in AI agents.

\begin{figure}[h!]
\centering
%%%\includegraphics[width = 14 cm]{vision_objective_aim.pdf}
\caption{\small Embedded relationships of broader vision, long-term objective, and present project}
\label{fig:visionobjective}
\end{figure}

